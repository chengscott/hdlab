\documentclass{article}
\usepackage{CJKutf8, indentfirst, graphicx, subfigure}
\begin{document}
\begin{CJK}{UTF8}{bsmi}
\title{硬體設計與實驗 Lab7 Report}
\author{104021219 鄭余玄}
\date{}
\maketitle
\section{實做過程}
這次 lab7 主要是參考助教 musicbox 的範例程式碼去略做修改,再加上助教已經提供了哆拉A夢和櫻桃小丸子的 Music.v,所以就可以很迅速的完成。BEAT\_FREQ 我是依照 Speed 訊號去選擇需要的速度,而 duty 則是按照 Quality 訊號去選擇不同的佔空比。 Repeat 訊號則是在 PlayerCtrl 這個 module 中處理,當音樂跑到最後時,如果有 Repeat 訊號則從頭播放,若沒有則設 ibeat 為 511(超過兩首音樂的 BEATLEAGTH 即可)。 此外,靜音我是利用 Mute 訊號用 mux 選擇,如果有靜音,pmod\_1就輸出兩萬赫茲,因此 pmod\_4 一直都是輸出 1。

Play、Stop、Music 訊號,我的用法比較特別,沒有使用助教建議的 FSM 寫法,而是把這幾種訊號當作 reset 訊號。 像是 Stop 訊號 or 上 Music 訊號就是第一個 PlayerCtrl 的 reset;而 Stop 訊號 or 上 ~Music 訊號就是第二個的 reset。因為 Stop 訊號其實等效讓音樂從頭開始播放,而 Music 也是會讓切換到的音樂從開始。 Music 訊號同時是用來選擇輸出的 freq。Start 和 Stop 訊號也類似以往的處理方式,每當訊號有改變時, enable 訊號就會改變,而這個訊號則是接到兩個 PWM\_gen 的 reset,讓音樂可以達到播不播放的效果。

\section{學到的東西及遇到的困難}
這次學到的就是 pmod 的使用,以及如何控制音樂的輸出。像是樂譜轉成一個個音符的頻率,助教就有提供一些小撇步,還有學到佔空比等概念。這次 lab 基本上沒有遇到特別的困難,主要只是害怕耳機燒掉而已。

\section{想對老師或助教說的話}
覺得這個 lab 很有趣!也很感謝助教幫我們打好音樂的 module了!

\end{CJK}
\end{document}
